%%%%%%%%%%%%%%%%%%%%%%%%%%%%%%%%%%%%%%%%%%%%%%%%%%%%%%%%%%%%%%%%%%%%%%
%%  dissertation.tex, to be compiled with latex2e.                   %
%%  16 April 2012                                                    %
%%%%%%%%%%%%%%%%%%%%%%%%%%%%%%%%%%%%%%%%%%%%%%%%%%%%%%%%%%%%%%%%%%%%%%
%%                                                                   %
%%  Writing a Doctoral Dissertation with LaTeX at                    %
%%           Georgia State University                                %
%%                                                                   %
%%  (Running this ``template'' will generate the documentation.)     %
%%                                                                   %
%%%%%%%%%%%%%%%%%%%%%%%%%%%%%%%%%%%%%%%%%%%%%%%%%%%%%%%%%%%%%%%%%%%%%%

\documentclass[12pt,gsu,online,openany,hidelinks]{gsudiss}
% Remove the option "online" to produce a print version (page number at the center of the footer)
\printdraft{\textcolor{gray}\small DRAFT}    % In order to use this
                                   % command, you have to enable "drafts"
                                   % in the option of the gsudiss
                                   % class, otherwise it does
                                   % nothing. This prints the word
                                   % "DRAFT" in gray color in the header of your
                                   % dissertation. You can go wild
                                   % here if you want. Just make sure
                                   % you disable the "drafts" option
                                   % before final printing.

\usepackage{natbib}
% \usepackage{subfigure}                % To format bibliographies.
\setlength{\bibsep}{0pt}           % Necessary for bib entries to have
                                   % correct line spacing.

\usepackage[hidelinks]{hyperref}
\hypersetup{
    colorlinks=false,
    pdfborder={0 0 0},
}

\usepackage{lscape}
\usepackage{array}
\usepackage{deluxetable}
\usepackage{longtable,ltcaption}
\usepackage{float}
\usepackage[caption = true]{subfig}
% The ltcaption package supports \CaptionLabelFont & \CaptionTextFont
% introduced by the NTG document classes
\renewcommand\CaptionLabelFont{\normalsize}
\renewcommand\CaptionTextFont{\normalsize}

\usepackage{amsmath,amsthm,amsfonts,amsopn,amssymb} % Some nice math packages.
\usepackage{eucal}                 % Euler fonts for equations.
\usepackage{verbatim}              % Allows quoting source with commands.
\usepackage{graphicx}              % For powerful manipulation of figures.
\usepackage{ctable}                % My preference table package.
\usepackage[overlay]{textpos}      % Put stuff anywhere, I mean anywhere ...
\usepackage{pstricks}              % Draw stuff especially on top figures etc...
\usepackage{afterpage}             % Useful for absolute placement of figures and tables.
\usepackage{longtable}             % for 'longtable' environment
\usepackage{pdflscape}             % for 'landscape' environment
\usepackage{fancyvrb}              % for verbatim environments (see usercommands.tex)

\usepackage{float}
\floatstyle{boxed}
\newfloat{code}{h}{ext}
\floatname{code}{Code}

\usepackage{atbeginend}            % Modify space before and after
                                   % equations. These are my preferences.
\AfterBegin{equation}{\addtolength{\abovedisplayskip}{-0.5\baselineskip}}
\BeforeEnd{equation}{\addtolength{\belowdisplayskip}{-0.5\baselineskip}}
\AfterBegin{equation*}{\addtolength{\abovedisplayskip}{-0.5\baselineskip}}
\BeforeEnd{equation*}{\addtolength{\belowdisplayskip}{-0.5\baselineskip}}


%%%%%%%%%%%%%%%%%%%%%%%%%%%%%%%%%%%%%%%%%%%%%%%%%%%%%%%%%%%%%%%%%%%%%%
%           Fill out some dissertation info first.                   %
%%%%%%%%%%%%%%%%%%%%%%%%%%%%%%%%%%%%%%%%%%%%%%%%%%%%%%%%%%%%%%%%%%%%%%

\author(FirstName LastName) % Your name.

\graduationyear(Year)
\graduationmonth(Month) % The Month and Year here must be the month and year of your graduation—not the month and year of your submission or upload.

\title(Manuscript Title) % The title of your dissertation/thesis.

%%%%%%%%%%%%%%%%%%%%%%%%%%%%%%%%%%%%%%%%%%%%%%%%%%%%%%%%%%%%%%%%
% Information of your committee members. I have five members as an example below.
% Only the first and last name of your committee members should be listed here. Please do not include their titles. In other words, do not list “Dr.” or “Ph.D.” here. 

\committeeChair(FirstName LastName1) % Committee CHAIR's First Name Last Name. This has to be non-empty to compile correctly.

\committeeCoChair() % Committee CO-CHAIR's First Name Last Name. This is useful if you have two advisors. If you do not have a co-chair, you can leave this field empty.

\committee(FirstName LastName2)(FirstName LastName3)(FirstName LastName4)(FirstName LastName5) % Other Committee Members. There has to be at least one to compile correctly. Just add more parentheses for more committee members.


%%%%%%%%%%%%%%%%%%%%%%%%%%%%%%%%
% Custom Options for the table of contents

\titlesize(12)(12)                           % This is for changing
                                             % the default font size
                                             % for your title (for just
                                             % the cover page). The
                                             % first argument is the
                                             % font size, the second
                                             % is the line spacing for
                                             % long tiles that wrap to
                                             % more than one line.
                                             % Default is equivalent
                                             % to the \LARGE command,
                                             % which is roughly (22)(22).

% You can set the fonts for each section of the TOC using options in the tocloft package
\renewcommand{\cftfigfont}{Figure\ }
\renewcommand{\cfttabfont}{Table\ }
\renewcommand{\cftchapfont}{\bfseries}  % Set the font for the chapters in TOC
\renewcommand{\cftsecfont}{\normalfont} % Set the font for the sections in TOC, e.g., you can change it to \renewcommand{\cftsecfont}{\bfseries}
\renewcommand{\cftsubsecfont}{\itshape}  % Set the font for the subsections in TOC, e.g., you can change it to \renewcommand{\normalfont}
\renewcommand{\cftsubsubsecfont}{\itshape}  % Set the font for the subsubsections in TOC
\renewcommand{\cftparafont}{\mdseries}  % Set the font for the paragraphs in TOC
\setcounter{tocdepth}{2}      % The depth of table of contents, the setting of 2 means it only goes to subsections, you can change it (e.g. 3 is to include subsubsections).


\renewcommand{\topfraction}{0.85}        % These modify figure placement on
\renewcommand{\bottomfraction}{0.85}     % the page and various other space
\renewcommand{\textfraction}{0.10}       % requirements for figures.
\renewcommand{\floatpagefraction}{0.80}  % These 5 lines are not
\renewcommand{\arraystretch}{0.5}        % necessary but I think its better
                                         % than latex default.

\setlength{\tabcolsep}{3pt}        % shrink column spacing so your tables 
                                   % can be wider (yay Todd Tables)
%\setlength{\LTcapwidth}{\textwidth}% so your rotated, normal-sized
                                   % longtable titles won't wrap oddly.


\clubpenalty=1000                  % Make Latex try hard to fix
\widowpenalty=1000                 % "stray lines" in paragraphs,
                                   % i.e. paragraph that begin at the
                                   % last line of a page, or end with
                                   % the last line on the following
                                   % page. This looks silly.
\raggedbottom

\settocname{TABLE OF CONTENTS}              % Set the "Table of Contents"
                                   % name. This is the default. You
                                   % can use "Table of Contents" for example.

\setlofname{LIST OF FIGURES}               % Change the name from 'List of
                                   % Figures'. Use whatever you wish.

\setlotname{LIST OF TABLES}                % Change the name from 'List of
                                   % Tables'. Use whatever suit your
                                   % fancy.

\settocbibname{REFERENCES}         % Change the name from
                                   % 'Bibliography'. Change it back if
                                   % you feel like it.

\setloaname{LIST OF ABBREVIATIONS}
                                   % I Changed the name from 'List of
                                   % Abbreviations'. Use any name that
                                   % makes sense here. If you don't
                                   % have an "abbreviations.tex" file,
                                   % this command will do nothing.

\setfigname{Figure\ }               % Set the caption labels for figures.

\settabname{Table\ }                % Set the caption labels for tables.

\setcapfont{pnc}                   % Set the caption font for
                                   % both tables and figures.

\chapternumsize{\normalsize}            % You can use any standard latex sizes here.
\chapterheadsize{\normalsize}           % You can use any standard latex sizes here.
\chaptertitlesize{\normalsize}           % You can use any standard latex sizes here.
                                   % These defaults look good to me.

\beforechapterheadname{CHAPTER}         % Optional text to put in front of
                                   % the chapter number.
\afterchapterheadname{}          % Optional text to put after the
                                   % chapter number. The default
                                   % looks like this: --1--. Of course
                                   % you can change this to any
                                   % format, for e.g. $\sim$

\chapterheadpos{center}            % You can use 'right', 'left',
                                   % 'center'.

\chaptertitlepos{center}           % You can use 'right', 'left',
                                   % 'center'

\chapterheadverticalspace{-1em}     % The space between the Chapter head
                                   % and the top of page. This distance
                                   % is not absolute, but relative to
                                   % the parameters set by the
                                   % geometry package. Play around
                                   % with this number to suit your needs.

\chapterbetweentitlespace{-1.em}   % The space between the chapter head
                                   % and the title head.

\titleheadverticalspace{2em}       % The space between the title head
                                   % and the text.

\sectiontitlesize{\normalsize}          % This is obvious.
\sectiontitlepos{left}             % Obvious.

\sectiontitleverticalspace{1em}    % The space between the section head
                                   % and the text.

\subsectiontitlesize{\normalsize}       % Obvious.
\subsectiontitlepos{left}          % Obvious.

\subsectiontitleverticalspace{0.5em} % You get the idea...

\subsubsectiontitlesize{\normalsize}
\subsubsectiontitlepos{left}

\subsubsectiontitleverticalspace{0.5em}

\sepabbrev{7em}                    % The space between the abbreviation
                                   % lists, that is, if you have
                                   % one. Has to be >= 5em.

\prettify{pnc}
        % There are many places that's 
                                   % required by the guidelines so
                                   % you can customize. This file
                                   % contains many the options that
                                   % you can set (or just ignore).

\newcommand\arcdeg{\mbox{\ensuremath{^\circ}}}%
\newcommand\arcmin{\mbox{\ensuremath{^\prime}}}%
\newcommand\arcsec{\mbox{\ensuremath{^{\prime\prime}}}}%
\newcommand{\point}{\mbox{\ensuremath{\!\!.}\thinspace}}
\newcommand{\minusone}{\ensuremath{^{-1}}}
\newcommand{\minustwo}{\ensuremath{^{-2}}}
\newcommand{\minusthree}{\ensuremath{^{-3}}}
\newcommand{\minusfive}{\ensuremath{^{-5}}}
\newcommand{\plusthree}{\ensuremath{^{3}}}
\newcommand{\plusfive}{\ensuremath{^{5}}}
\newcommand{\kms}{\mbox{\ km\,s\ensuremath{^{-1}}}}
\newcommand{\fig}{Figure~}
\newcommand{\figs}{Figures~}
\newcommand{\tab}{Table~}
\newcommand{\tabs}{Tables~}
\newcommand{\eqn}{Equation~}
\newcommand{\eqns}{Equations~}
\newcommand{\vracc}{\mbox{$\mathrm{v=kr}$}\xspace}
\newcommand{\vrdec}{\mbox{$\mathrm{v=v_{max}-k^{'}(r-r_t)}$}\xspace}
\newcommand{\vrootracc}{\mbox{$\mathrm{v=k_{1}\sqrt r}$}\xspace}
\newcommand{\vrootrdec}{\mbox{$\mathrm{v=v_{max}-k_{2}\sqrt{r-r_t}}$}\xspace}
\newcommand{\rlaw}{\mbox{$r\ $law}\xspace}
\newcommand{\rootrlaw}{\mbox{$\sqrt r\ $law}\xspace}
\newcommand{\resolvingpower}{\mbox{$\lambda/\Delta\lambda$}\xspace}
\newcommand{\OIII}{\mbox{[\acs{O3}]}\xspace}
\newcommand{\solarmass}{\mbox{\ M\ensuremath{_{\odot}}}}
\newcommand{\arcpt}{${{\lower3pt\hbox{$^{\prime\prime}$}}\atop{\raise4pt\hbox{.}}}$}
\newcommand{\msun}{$M_\odot$}               % You can put some self-defined
                                   % commands in this file like
                                   % math abbreviations, etc.


\begin{document}

%%%%%%%%%%%%%%%%%%%%%%%%%%%%%%%%%%%%%%%%%%%%%%%%%%%%%%%%%%%%%%%%%%%%%%
%                 The document starts here.                          %
%%%%%%%%%%%%%%%%%%%%%%%%%%%%%%%%%%%%%%%%%%%%%%%%%%%%%%%%%%%%%%%%%%%%%%

%%%%%%%%%%%%%%%%%%%% The front matter of your document %%%%%%%%%%%%%%%
% The front matters include copyright page (required), approval page (required), dedication (optional), acknowledgments (optional), table of contents (required), list of tables(required and generated automatically if tables used), list of figures (required and generated automatically if figures used), list of abbreviations (optional), and preface (optional) 


% Tile and abstract pages are required. To edit your title and abstract pages, go to `./FrontMatters/TitleAbstract.tex' file.
\pagestyle{empty}
\begin{center}
Manuscript Title % Your title should be in title caps.

%%%%%%%%%%%%%%%%
% If it takes more than two lines, it must be double-spaced, for example,

% Manuscript Title line 1
% \vspace*{.1in}
% Manuscript Title line 2
%%%%%%%%%%%%%%%%

\vspace{1.4in}
by\\
\vspace{1.4in}
First and Last Name\\ %Your name should be in title caps.
\vspace{1.4in}
Under the Direction of Committee Chair's Name, MA/Ph.D. \\ %List your supervisor's first and last name followed by their letter credentials. Do not put their title before their name.
\vspace{1.4in}
A Thesis/Dissertation Submitted in Partial Fulfillment of the Requirements for the Degree of\\  % Choose either thesis or dissertation and delete the other.
\vspace{.2in}
Level and Degree Title \\ % Your degree does not include your specialization. The following are your options: Master of Arts, Master of Science, or Doctor of Philosophy. Do NOT list your department/field here.
\vspace{.2in}
in the College of Arts and Sciences \\
\vspace{.2in}
Georgia State University \\
\vspace{.2in}
Year

\end{center}

\pagebreak 


\begin{center}
    ABSTRACT\\ 
\end{center}

\doublespacing
The abstract paragraph is mandatory. Start the abstract paragraph here. Double-space this paragraph. Limit the abstract of a thesis to 150 words. Limit the abstract of a dissertation to 350 words. ``Your abstract will not be accepted if it exceeds the limit by even one word.''

\begin{singlespace}
\vfill   
\vspace{0.5in}
\noindent INDEX WORDS:
\hspace{0in}
\parbox[t]{4.5in}{
Sample keyword, Sample keyword, Sample keyword}  %It is recommended that you list six keywords. Please use my capitalization here as an example. Should be single-spaced.
\end{singlespace} 

\frontmatter % This line is supposed to be here.

% The title page is required. To edit your name and year, please go to \author()
\copyrightpage

% The approval page is required. To edit the committee and other information, please go to \committeeChair(), \committeeCoChair(), \committee(), etc.
\approvalpage

% The dedication page is optional. You can write your dedication in `./FrontMatters/dedication.tex' file. Delete that file if you don't want to include it. This command does nothing if you don't have a `dedication.tex' file.
\dedicationpage

% The acknowledgment page is optional. You can write your acknowledgment in `./FrontMatters/acknowledgment.tex' file. Delete that file if you don't want to include it. This command does nothing if you don't have an `acknowledgment.tex' file.
\acknowledgmentpage

% Table of Contents will be automatically generated and placed here.
\tableofcontents

% List of Tables will be automatically generated if you had made proper table captions.
\listoftables

% List of Figures will be automatically generated if you had made proper figure captions.
\listoffigures

% The list of abbreviations is optional. List of Abbreviations will be automatically generated if you had made any in `./FrontMatters/abbreviations.tex' file following the style of the "Acronym". See that file for example usage. Delete that file if you don't want to include it. This command does nothing if you don't have an `abbreviations.tex' file.
\listofabbreviations


%%%%%%%%%%%%%%%%%%%% The main chapters %%%%%%%%%%%%%%%%%%%%%

\pagestyle{plain}
\mainmatter                        % Main chapters starts here

% The chapter order is determined by the order of the \include commands below.

\chapter{Introduction}
\label{chap:introduction}

\section{Your Section Title Here}
	You should probably introduce some stuff here.

\subsection{Your Subsection Title Here}
You should probably introduce some stuff here.

You should look at the gsudiss.cls file first when you want to change things in how the document is laid out. 

You can refer to your other chapters like this: \chap\ref{chap:chapter_2}. Since Appendix is not numbered, you can refer to it like this: \hyperref[chap:appendix]{Appendix}.

\subsubsection{Your Subsubsection Title Here}
This is the content for the subsubsection.

\chapter{CHAPTER 2 TITLE}

Hooray for Chapter 2!!!

Sample Figures and Tables below.

And as an example of citing things, I'm going to cite one of our paper~\cite{LAC_first}. 

See \verb|bibliography.bib| for doing references.

\begin{figure*}[ht]
	\includegraphics[width=\textwidth]{./Plots/nature.jpg}
	\caption{An individual figure!}
\end{figure*}
        
\begin{figure*}[ht]
	\subfloat[\label{fig:HD8538_ellplot}]{\includegraphics[width=.48\textwidth]{./Plots/nature.jpg}} 
	\subfloat[\label{fig:HD8538_phot}]{\includegraphics[width=.48\textwidth]{./Plots/rocks.jpg}} \\
	\subfloat[\label{fig:HD8538_vis}]{\includegraphics[width=.48\textwidth]{./Plots/rocks.jpg}}
	\subfloat[\label{fig:HD8538_HRD}]{\includegraphics[width=.48\textwidth]{./Plots/nature.jpg}}
	\caption{Multiple figures!}
\end{figure*}


\begin{landscape}
\begin{longtable}{cccccccccccccc}
\label{tab:disk}\\
\caption{Insert Table Caption here}\\
\hline\endhead  % header material
\hline\endfoot  % footer material
\hline
Blah & Blah & Blah \\
\hline
Stuff & Things & etc. \\
\nodata & \nodata & \nodata \\
\end{longtable}
\end{landscape}

\chapter{CHAPTER 3 TITLE}

Here is an example of code snippet:
\begin{code}
	print("Hello World!")
	for i in range(10):
		print(i)
\end{code}

If you want some example text, here is some:
\begin{example}
	Your example here
\end{example}


%%%%%%%%%%%%%%%%%%%% The back matters %%%%%%%%%%%%%%%%%%%%%

\appendix

% Use this file if you have only one appendices. If you have more than one appendix, use Appendices.tex instead
% If you want to write multiple sections in the appendix, it's better also use Appendices.tex 

\chapter*{APPENDIX}
\label{chap:appendix}
\addcontentsline{toc}{chapter}{APPENDIX}

This is the appendix! You can actually have a ``phantom'' section in the appendix like the following and refer to it in your text like this: \hyperref[sec:appendix_phantom_section]{Appendix}.

\phantomsection
\section*{A section title in the appendix}
\label{sec:appendix_phantom_section}

However, you probably want to use \verb|Appendices.tex| instead of this file if you really have multiple appendices instead of separating phantom sections.
 % Use this file if you have only one appendix.
% % Use this file if you have more than one appendices. If you have only one appendix, use Appendix.tex instead.

\chapter*{APPENDICES}
\addcontentsline{toc}{chapter}{APPENDICES}

\section{Something}
This is the appendix!

\subsection{First subsection}
Explain some more details here.


\section{Something Else}
Another appendix! % Use this file if you have more than one appendices.


% The bibliography starts here.
% Format the entries according to your department or discipline’s choice of style manual.
\bibliographystyle{plain}           % Please learn to use the
                                    % formatting of Latex's Bibtex. It
                                    % will make your life easier.
\bibliography{mybibliography}       % "mybibliography.bib" contains some
                                    % reference examples.

%%%%%%%%%%%%%%%%%%%%%%%%%%%%%%%%%%%%%%%%%%%%%%%%%%%%%%%%%%%%%%%%%%%%%%
%                  The document ends here.                           %
%%%%%%%%%%%%%%%%%%%%%%%%%%%%%%%%%%%%%%%%%%%%%%%%%%%%%%%%%%%%%%%%%%%%%%

\end{document}