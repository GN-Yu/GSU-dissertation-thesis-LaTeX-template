%%%%%%%%%%%%%%%%%%%%%%%%%%%%%%%%
% Custom Options for the table of contents

% You can set the fonts for each section of the TOC using options in the tocloft package
\renewcommand{\cftfigfont}{Figure\ }
\renewcommand{\cfttabfont}{Table\ }
\renewcommand{\cftchapfont}{\bfseries}  % Set the font for the chapters in TOC
\renewcommand{\cftsecfont}{\normalfont} % Set the font for the sections in TOC, e.g., you can change it to \renewcommand{\cftsecfont}{\bfseries}
\renewcommand{\cftsubsecfont}{\itshape}  % Set the font for the subsections in TOC, e.g., you can change it to \renewcommand{\normalfont}
\renewcommand{\cftsubsubsecfont}{\itshape}  % Set the font for the subsubsections in TOC
\renewcommand{\cftparafont}{\mdseries}  % Set the font for the paragraphs in TOC
\setcounter{tocdepth}{2}      % The depth of table of contents, the setting of 2 means it only goes to subsections, you can change it (e.g. 3 is to include subsubsections).


\renewcommand{\topfraction}{0.85}        % These modify figure placement on
\renewcommand{\bottomfraction}{0.85}     % the page and various other space
\renewcommand{\textfraction}{0.10}       % requirements for figures.
\renewcommand{\floatpagefraction}{0.80}  % These 5 lines are not
\renewcommand{\arraystretch}{0.5}        % necessary but I think its better
                                         % than latex default.

\setlength{\tabcolsep}{3pt}        % shrink column spacing so your tables 
                                   % can be wider (yay Todd Tables)
%\setlength{\LTcapwidth}{\textwidth}% so your rotated, normal-sized
                                   % longtable titles won't wrap oddly.


\clubpenalty=1000                  % Make Latex try hard to fix
\widowpenalty=1000                 % "stray lines" in paragraphs,
                                   % i.e. paragraph that begin at the
                                   % last line of a page, or end with
                                   % the last line on the following
                                   % page. This looks silly.
\raggedbottom

\settocname{TABLE OF CONTENTS}              % Set the "Table of Contents"
                                   % name. This is the default. You
                                   % can use "Table of Contents" for example.

\setlofname{LIST OF FIGURES}               % Change the name from 'List of
                                   % Figures'. Use whatever you wish.

\setlotname{LIST OF TABLES}                % Change the name from 'List of
                                   % Tables'. Use whatever suit your
                                   % fancy.

\settocbibname{REFERENCES}         % Change the name from
                                   % 'Bibliography'. Change it back if
                                   % you feel like it.

\setloaname{LIST OF ABBREVIATIONS}
                                   % I Changed the name from 'List of
                                   % Abbreviations'. Use any name that
                                   % makes sense here. If you don't
                                   % have an "abbreviations.tex" file,
                                   % this command will do nothing.

\setfigname{Figure\ }               % Set the caption labels for figures.

\settabname{Table\ }                % Set the caption labels for tables.

\setcapfont{pnc}                   % Set the caption font for
                                   % both tables and figures.

\chapternumsize{\normalsize}            % You can use any standard latex sizes here.
\chapterheadsize{\normalsize}           % You can use any standard latex sizes here.
\chaptertitlesize{\normalsize}           % You can use any standard latex sizes here.
                                   % These defaults look good to me.

\beforechapterheadname{CHAPTER}         % Optional text to put in front of
                                   % the chapter number.
\afterchapterheadname{}          % Optional text to put after the
                                   % chapter number. The default
                                   % looks like this: --1--. Of course
                                   % you can change this to any
                                   % format, for e.g. $\sim$

\chapterheadpos{center}            % You can use 'right', 'left',
                                   % 'center'.

\chaptertitlepos{center}           % You can use 'right', 'left',
                                   % 'center'

\chapterheadverticalspace{-1em}     % The space between the Chapter head
                                   % and the top of page. This distance
                                   % is not absolute, but relative to
                                   % the parameters set by the
                                   % geometry package. Play around
                                   % with this number to suit your needs.

\chapterbetweentitlespace{-1.em}   % The space between the chapter head
                                   % and the title head.

\titleheadverticalspace{2em}       % The space between the title head
                                   % and the text.

\sectiontitlesize{\normalsize}          % This is obvious.
\sectiontitlepos{left}             % Obvious.

\sectiontitleverticalspace{1em}    % The space between the section head
                                   % and the text.

\subsectiontitlesize{\normalsize}       % Obvious.
\subsectiontitlepos{left}          % Obvious.

\subsectiontitleverticalspace{0.5em} % You get the idea...

\subsubsectiontitlesize{\normalsize}
\subsubsectiontitlepos{left}

\subsubsectiontitleverticalspace{0.5em}

\sepabbrev{7em}                    % The space between the abbreviation
                                   % lists, that is, if you have
                                   % one. Has to be >= 5em.

\prettify{pnc}
