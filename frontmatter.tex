%%%%%%%%%%%%%%%%%%%% The front matter of your document %%%%%%%%%%%%%%%%%%%%

\frontmatter

% \certifypage                 % Produces the certify page, if this option is
%                              % set in the class file.



%\copyrightpage               % Produces the copyright page.

%%%%%%%%%%%%%%
%Below is the copy right page
%%%%%%%%%%%%%%

\thispagestyle{empty}
\vfill
\begin{center}
\setstretch{0.9}
\vspace*{\fill}
Copyright by\\
Full Legal Name\\ %Your name must be in title caps here.
Year\\ %The current Year
\end{center}
%%%%%%%%%%%%%%%%%%%%%%%%%%%%%%%%%%%

\clearpage
\approvalpage                % To edit the Committee, First Name & Last Name, please go to file main.tex
\clearpage
\dedicationpage              % The dedication page is optional.
\clearpage                             % This command does nothing if you don't
                             % have a `dedication.tex' file, otherwise
                             % the file is included in the frontmatter.

\acknowledgmentpage          % The acknowledge page is optional.
\clearpage                             % This command does nothing if you don't
                             % have an `acknowledgment.tex' file, otherwise
                             % the file is included in the frontmatter.

\tableofcontents             % Table of Contents will be automatically
\clearpage                             % generated and placed here.

\listoftables                % List of Tables will be automatically
\clearpage                             % generated if you had made proper table captions.

\listoffigures               % List of Figures will be automatically
\clearpage                             % generated if you had made proper figure captions.

\listofabbreviations         % List of Abbreviations will be
\clearpage                             % automatically generated if you had made any,
                             % following the style of the "Acronym"
                             % package. See my "abbreviation.tex" file
                             % for example usage. If you don't have
                             % this file, the command does nothing.
